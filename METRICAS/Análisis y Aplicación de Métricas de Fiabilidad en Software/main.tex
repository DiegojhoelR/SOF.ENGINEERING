\documentclass[a4,10pt]{article}
\usepackage[margin=1in]{geometry}
\usepackage{fancyhdr}
\usepackage{graphicx}
\usepackage{cancel}
\usepackage[english]{babel}
\usepackage{hyperref}
\usepackage{listings}
\usepackage[
backend=biber,
style=ieee,
]{biblatex}
\usepackage{amsmath}
\usepackage{xcolor}

\geometry{a4paper, margin=1in}

% Definir una nueva función para agregar imágenes
\newcommand{\agregarimagen}[2]{%
    \begin{figure}[htbp]%
        \centering%
        \includegraphics[width=\linewidth,height=\textheight,keepaspectratio]{#1}%
        \caption{#2}%
        \label{fig:#1}%
    \end{figure}%
}

\definecolor{codegreen}{rgb}{0,0.6,0}
\definecolor{codegray}{rgb}{0.5,0.5,0.5}
\definecolor{codepurple}{rgb}{0.58,0,0.82}
\definecolor{backcolour}{rgb}{0.95,0.95,0.92}

\lstdefinestyle{mystyle}{
    backgroundcolor=\color{backcolour},   
    commentstyle=\color{codegreen},
    keywordstyle=\color{magenta},
    numberstyle=\tiny\color{codegray},
    stringstyle=\color{codepurple},
    basicstyle=\footnotesize,
    breakatwhitespace=false,         
    breaklines=true,                 
    captionpos=b,                    
    keepspaces=true,                 
    numbers=left,                    
    numbersep=5pt,                  
    showspaces=false,                
    showstringspaces=false,
    showtabs=false,                  
    tabsize=2
}

\lstset{style=mystyle}

\addbibresource{references.bib}

\pagestyle{fancy}
\fancyhead[LO,L]{FINESI}
\fancyhead[CO,C]{Ingeniería de Software I}
\fancyhead[RO,R]{\today}
\fancyfoot[LO,L]{Mamani Romero Diego Jhoel}
\fancyfoot[CO,C]{}
\fancyfoot[RO,R]{Page. \thepage}
\renewcommand{\headrulewidth}{0.4pt}
\renewcommand{\footrulewidth}{0.4pt}

\begin{document}

\section{Introducción}
La modularidad es un principio fundamental en el diseño de software que tiene un impacto significativo en atributos de calidad externa como la reutilización, mantenibilidad y comprensión del software \cite{xiang2019}.

\section{Definición y Tipos}

\subsection{Definición de Modularidad}
La modularidad en software se define como la medida en que un sistema de software está compuesto de componentes independientes o módulos que pueden ser desarrollados y modificados de manera separada \cite{xiang2019}. Las métricas de modularidad ayudan a evaluar esta capacidad, identificando problemas potenciales y planificando mejoras \cite{rahardjo2011}.

\subsection{Principales Métricas de Modularidad}
\begin{itemize}
    \item \textbf{Acoplamiento (Coupling)}: Mide la dependencia entre módulos. Menor acoplamiento indica una mejor modularidad.
    \item \textbf{Cohesión (Cohesion)}: Mide la fuerza con la que los elementos dentro de un módulo están relacionados entre sí. Mayor cohesión indica una mejor modularidad.
    \item \textbf{Índice de Modularidad (Modularity Index)}: Una métrica cuantitativa que analiza características como tamaño, complejidad, cohesión y acoplamiento para evaluar la modularidad de proyectos de software de código abierto \cite{rahardjo2011}.
\end{itemize}

\section{Aplicaciones y Limitaciones}
Las métricas de modularidad se utilizan para varios propósitos importantes en el desarrollo y mantenimiento de software \cite{xiang2019}:

\subsection{Aplicaciones}
\begin{itemize}
    \item \textbf{Evaluación de la Calidad del Diseño}: Permiten evaluar la calidad del diseño del software en términos de modularidad.
    \item \textbf{Mantenimiento del Software}: Facilitan la identificación de módulos problemáticos que requieren refactorización.
    \item \textbf{Reutilización del Software}: Ayudan a identificar módulos que pueden ser reutilizados en otros proyectos.
    \item \textbf{Evolución del Software}: Asisten en la planificación de la evolución del software, asegurando que se mantenga una buena modularidad a lo largo del tiempo \cite{xiang2019}.
\end{itemize}

\subsection{Limitaciones}
A pesar de sus beneficios, las métricas de modularidad tienen ciertas limitaciones \cite{rahardjo2011}:
\begin{itemize}
    \item \textbf{Evaluación Incompleta}: Muchas métricas solo evalúan un aspecto de la modularidad, como el acoplamiento o la cohesión, pero no ambos.
    \item \textbf{Datos Limitados}: La precisión de las métricas depende de la calidad y cantidad de datos disponibles.
    \item \textbf{Variabilidad de Contexto}: Las métricas pueden no ser directamente comparables entre diferentes proyectos debido a las variaciones en los contextos de desarrollo y uso.
\end{itemize}

\section{Código en Python}
A continuación se presenta un ejemplo de código en Python para calcular las métricas de acoplamiento y cohesión utilizando datos simulados:

\begin{lstlisting}[language=Python, caption=Cálculo de Métricas de Modularidad en Python]
import numpy as np

# Datos simulados de acoplamiento y cohesión
acoplamiento = np.array([0.2, 0.3, 0.1, 0.4, 0.5])
cohesion = np.array([0.8, 0.7, 0.9, 0.6, 0.5])

# Cálculo de métricas
mean_acoplamiento = np.mean(acoplamiento)
mean_cohesion = np.mean(cohesion)

print(f"Acoplamiento Promedio: {mean_acoplamiento:.2f}")
print(f"Cohesión Promedio: {mean_cohesion:.2f}")
\end{lstlisting}

Este código en Python simula datos de acoplamiento y cohesión, y luego calcula las métricas promedio utilizando estos datos.

\section{Conclusión}
Las métricas de modularidad son esenciales para garantizar que el software esté bien diseñado y sea mantenible. A través de la implementación y análisis de estas métricas, se pueden identificar problemas, mejorar la calidad del diseño y asegurar la modularidad a largo plazo.

\printbibliography

\end{document}
