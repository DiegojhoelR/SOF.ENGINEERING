\documentclass[a4,10pt]{article}
\usepackage[margin=1in]{geometry}
\usepackage{fancyhdr}
\usepackage{graphicx}
\usepackage{cancel}
\usepackage[english]{babel}
\usepackage{hyperref}
\usepackage{listings}
\usepackage[
backend=biber,
style=ieee,
]{biblatex}
\usepackage{amsmath}
\usepackage{xcolor}

\geometry{a4paper, margin=1in}

% Definir una nueva función para agregar imágenes
\newcommand{\agregarimagen}[2]{%
    \begin{figure}[htbp]%
        \centering%
        \includegraphics[width=\linewidth,height=\textheight,keepaspectratio]{#1}%
        \caption{#2}%
        \label{fig:#1}%
    \end{figure}%
}

\definecolor{codegreen}{rgb}{0,0.6,0}
\definecolor{codegray}{rgb}{0.5,0.5,0.5}
\definecolor{codepurple}{rgb}{0.58,0,0.82}
\definecolor{backcolour}{rgb}{0.95,0.95,0.92}

\lstdefinestyle{mystyle}{
    backgroundcolor=\color{backcolour},   
    commentstyle=\color{codegreen},
    keywordstyle=\color{magenta},
    numberstyle=\tiny\color{codegray},
    stringstyle=\color{codepurple},
    basicstyle=\footnotesize,
    breakatwhitespace=false,         
    breaklines=true,                 
    captionpos=b,                    
    keepspaces=true,                 
    numbers=left,                    
    numbersep=5pt,                  
    showspaces=false,                
    showstringspaces=false,
    showtabs=false,                  
    tabsize=2
}

\lstset{style=mystyle}

\addbibresource{references.bib}

\pagestyle{fancy}
\fancyhead[LO,L]{FINESI}
\fancyhead[CO,C]{Ingeniería de Software I}
\fancyhead[RO,R]{\today}
\fancyfoot[LO,L]{Mamani Romero Diego Jhoel}
\fancyfoot[CO,C]{}
\fancyfoot[RO,R]{Page. \thepage}
\renewcommand{\headrulewidth}{0.4pt}
\renewcommand{\footrulewidth}{0.4pt}

\begin{document}

\section{Introducción}
La fiabilidad es un aspecto fundamental en la calidad del software. Se refiere a la capacidad del software para funcionar correctamente bajo condiciones específicas durante un período determinado \cite{reliably2022}.

\section{Definición y Tipos}

\subsection{Definición de Fiabilidad}
La fiabilidad de un software se define como la probabilidad de que un sistema funcione sin fallos durante un período específico en un entorno determinado \cite{reliably2022}. Las métricas de fiabilidad ayudan a medir esta capacidad, identificar problemas potenciales y planificar mejoras \cite{javatpoint2022}.

\subsection{Principales Métricas de Fiabilidad}
\begin{itemize}
    \item \textbf{MTTF (Mean Time to Failure)}: Tiempo promedio hasta que ocurre la primera falla.
    \item \textbf{MTTR (Mean Time to Repair)}: Tiempo promedio que se tarda en reparar una falla.
    \item \textbf{MTBF (Mean Time Between Failures)}: Tiempo promedio entre fallos sucesivos (MTBF = MTTF + MTTR).
    \item \textbf{FCR (Failure Count Rate)}: Tasa a la que ocurren fallos en el sistema.
    \item \textbf{FAR (Failure Analysis Rate)}: Tasa de análisis de fallos \cite{reliably2022}.
\end{itemize}

\section{Aplicaciones y Limitaciones}
Las métricas de fiabilidad se utilizan para varios propósitos importantes en el desarrollo y mantenimiento de software \cite{reliably2022}:

\subsection{Aplicaciones}
\begin{itemize}
    \item \textbf{Estimación de Costos y Tiempos}: Ayudan a estimar los costos y el tiempo necesarios para el desarrollo y mantenimiento del software.
    \item \textbf{Medición de Productividad}: Permiten medir la productividad del equipo de desarrollo y el rendimiento del software.
    \item \textbf{Monitoreo del Progreso del Proyecto}: Facilitan el seguimiento del progreso del proyecto, asegurando que se cumplan los objetivos de fiabilidad.
    \item \textbf{Mejora Continua}: Identifican áreas problemáticas que requieren optimización y mejoras, lo que contribuye a la mejora continua del software.
    \item \textbf{Planificación de Mantenimiento}: Proveen información crucial para planificar las actividades de mantenimiento preventivo y correctivo \cite{javatpoint2022}.
\end{itemize}

\subsection{Limitaciones}
A pesar de sus beneficios, las métricas de fiabilidad tienen ciertas limitaciones \cite{javatpoint2022}:
\begin{itemize}
    \item \textbf{Calidad del Código}: No indican la calidad ni la complejidad del código, por lo que deben complementarse con otras métricas.
    \item \textbf{Enfoque en la Cantidad}: Pueden fomentar un enfoque en la cantidad en lugar de la calidad, llevando a la escritura de código innecesario.
    \item \textbf{Datos Incompletos}: La precisión de las métricas depende de la completitud y exactitud de los datos recopilados.
    \item \textbf{Contexto Específico}: Las métricas pueden no ser directamente comparables entre diferentes proyectos debido a las variaciones en los contextos de desarrollo y uso.
\end{itemize}

\section{Código en Python}
A continuación se presenta un ejemplo de código en Python para calcular las principales métricas de fiabilidad utilizando datos simulados:

\begin{lstlisting}[language=Python, caption=Cálculo de Métricas de Fiabilidad en Python]
import numpy as np

# Datos simulados de tiempos entre fallos (en horas)
tiempos_entre_fallos = np.array([50, 60, 45, 70, 55, 65, 80, 75, 90, 85])
tiempos_de_reparacion = np.array([2, 3, 2, 4, 3, 3, 2, 4, 2, 3])

# Cálculo de métricas
MTTF = np.mean(tiempos_entre_fallos)
MTTR = np.mean(tiempos_de_reparacion)
MTBF = MTTF + MTTR
FCR = len(tiempos_entre_fallos) / np.sum(tiempos_entre_fallos)
FAR = len(tiempos_de_reparacion) / np.sum(tiempos_de_reparacion)

print(f"MTTF: {MTTF:.2f} horas")
print(f"MTTR: {MTTR:.2f} horas")
print(f"MTBF: {MTBF:.2f} horas")
print(f"FCR: {FCR:.2f} fallos/hora")
print(f"FAR: {FAR:.2f} análisis/hora")
\end{lstlisting}

Este código en Python simula tiempos entre fallos y tiempos de reparación, luego calcula las métricas de fiabilidad utilizando estos datos.

\section{Conclusión}
Las métricas de fiabilidad son esenciales para garantizar que el software funcione de manera efectiva y eficiente. A través de la implementación y análisis de estas métricas, se pueden identificar problemas, mejorar la calidad del software y asegurar su fiabilidad a largo plazo.

\printbibliography

\end{document}
